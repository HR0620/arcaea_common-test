\RequirePackage{plautopatch}
\documentclass[paper=a4paper,fontsize=11pt,jafontscale=0.9247,lang=ja]{jlreq}
\usepackage{bxpapersize}
\usepackage{fancyhdr}
\usepackage{fancybox,ascmac}
\usepackage{url}
\usepackage[autostyle]{csquotes}
\usepackage[japanese]{babel}
\usepackage{geometry}
\usepackage{array}
\usepackage{tabularx}
\usepackage{tikz}
\usepackage{amsmath}
\usepackage{amssymb}
\usepackage{enumitem}
\usepackage{etoolbox}
\usepackage{tcolorbox}
\tcbuselibrary{skins, breakable, listings}
\usepackage{listings}
\usepackage{listings}
\usepackage{hhline}
\usepackage{graphicx}
\usepackage{multicol} % 複数列表示用

% ページ余白の設定
\geometry{top=25mm, bottom=25mm, left=20mm, right=20mm}

% fancyhdrの設定
\setlength{\headheight}{18pt}
\setlength{\headsep}{12pt}
\setlength{\footskip}{20pt}

%===============================================
%  マークシート解答欄用マクロ
%===============================================
\newcounter{answercat}
\newcommand{\answerset}[1]{\setcounter{answercat}{#1}}
\newcommand{\katakana}[1]{\ifcase#1\or ア\or イ\or ウ\or エ\or オ\or カ\or キ\or ク\or ケ\or コ\or サ\or シ\or ス\or セ\or ソ\or タ\or チ\or ツ\or テ\or ト\or ナ\or ニ\or ヌ\or ネ\or ノ\or ハ\or ヒ\or フ\or ヘ\or ホ\or マ\or ミ\or ム\or メ\or モ\or ヤ\or ユ\or ヨ\or ラ\or リ\or ル\or レ\or ロ\or ワ\or ヲ\or ン\fi}

% 自動採番される解答欄(アイウエ...)
% 使用法: \autoanswer と書くとア、イ、ウ...と進む
\newcommand{\autoanswer}{%
  \stepcounter{answercat}%
  \doublebox{\ \bfseries\katakana{\value{answercat}}\ }%
}

% 指定した文字の解答欄
% 使用法: \setanswer{ア}
\newcommand{\setanswer}[1]{%
  \doublebox{\ \bfseries #1\ }%
}

% 解答群のボックス
% 使用法: \answergroup[列数]{ア}{選択肢の内容}
% デフォルトは1列
\newcommand{\answergroup}[3][1]{%
  \par\vspace{5pt}%
  \textbf{\doublebox{#2} の解答群}\par%
  \fbox{%
    \begin{minipage}{0.95\linewidth}%
      \vspace{5pt}%
      \ifnum #1=1
        #3
      \else
        \begin{multicols}{#1}
        #3
        \end{multicols}
      \fi
      \vspace{5pt}%
    \end{minipage}%
  }\par\vspace{10pt}%
}

% 表形式の解答群タイトル
% 使用法: \answertable{ア}{表の内容}
\newcommand{\answertable}[2]{%
  \par\vspace{5pt}%
  \textbf{\doublebox{#1} の選択肢}\par%
  \vspace{5pt}%
  \begin{center}
    #2
  \end{center}
  \par\vspace{10pt}%
}

% 四角で囲むだけのコマンド(共通テストの選択肢番号など)
% fancyboxのdoubleboxを上書きして、ユーザー好みの二重枠にする
\renewcommand{\doublebox}[1]{%
  \fbox{\fbox{\bfseries #1}}%
}

%===============================================
%  問題階層マクロ (Common Test Style)
%===============================================
% 第1問
\newcommand{\daimon}[1]{%
  \subsection*{#1}%
}

% [1], [2]
\newcommand{\syomon}[1]{%
  \subsubsection*{#1}%
}

% (1), (2)
\newcommand{\komon}[1]{%
  \textbf{#1}\quad%
}


%===============================================
%  情報関係基礎/情報I コード記述用スタイル
%===============================================
\lstdefinelanguage{DNCL}{
  morekeywords={表示する, もし, ならば, 実行し, そうでなければ, を, まで, 繰り返す, 繰り返し, 終わり, を, 代入する, 定義する, 返す, 関数},
  morecomment=[l]{\#},
  morestring=[b]",
  sensitive=true
}

\lstset{
  language=DNCL,
  basicstyle={\ttfamily\small},
  keywordstyle={\bfseries},
  commentstyle={\itshape},
  stringstyle={},
  frame=tlrb,
  framesep=5pt,
  numbers=left,
  numberstyle={\tiny},
  stepnumber=1,
  breaklines=true,
  showstringspaces=false,
  tabsize=2,
  captionpos=b
}

% tcolorboxを使ったコード表示用環境
\newtcblisting{dnclcode}[1][]{%
  colback=white,
  colframe=black,
  sharp corners,
  boxrule=0.5pt,
  listing only,
  listing options={language=DNCL},
  #1
}

%===============================================
%  ページスタイルの定義
%===============================================

% 表紙用スタイル(ヘッダー・フッターなし)
\fancypagestyle{coverpage}{
  \fancyhf{}
  \renewcommand{\headrulewidth}{0pt}
  \renewcommand{\footrulewidth}{0pt}
}

% 注意事項ページ用スタイル
\fancypagestyle{instructionpage}{
  \fancyhf{}
  \renewcommand{\headrulewidth}{0pt}
  \renewcommand{\footrulewidth}{0pt}
  \cfoot{\thepage}
}

% 科目ページ用スタイル(通常)
\fancypagestyle{exampage}{
  \fancyhf{}
  \lhead{\small 令和8年度試験}
  \chead{\small 【第1回 Arcaea共通テスト】}
  \renewcommand{\headrulewidth}{0.4pt}
  \cfoot{\small ―\ \thepage\ ―}
  \rfoot{\small (次頁へ続く)}
  \renewcommand{\footrulewidth}{0.4pt}
}

% 科目の最終ページ用スタイル
\fancypagestyle{exampageend}{
  \fancyhf{}
  \lhead{\small 令和8年度試験}
  \chead{\small 【第1回 Arcaea共通テスト】}
  \renewcommand{\headrulewidth}{0.4pt}
  \cfoot{\small ―\ \thepage\ ―}
  \rfoot{\small (試験問題は以上です)}
  \renewcommand{\footrulewidth}{0.4pt}
}

% 謝辞用スタイル
\fancypagestyle{acknowledgepage}{
  \fancyhf{}
  \renewcommand{\headrulewidth}{0pt}
  \renewcommand{\footrulewidth}{0pt}
}

%===============================================
%  科目開始コマンド
%===============================================
% 使い方: \beginsubject{科目名}
\newcommand{\beginsubject}[1]{%
  \newpage
  \pagestyle{exampage}%
  \begin{center}
    \vspace*{2em}
    {\LARGE\textbf{#1}}\\[0.5em]
    \rule{0.6\textwidth}{0.8pt}
  \end{center}
  \vspace{1em}
}

% 科目終了(最終ページにマーク)
% 使い方: そのセクションの最後で \finishsubject を呼ぶ
\newcommand{\finishsubject}{%
  \thispagestyle{exampageend}%
}

\begin{document}

%===============================================
%  表紙
%===============================================
\thispagestyle{coverpage}
\begin{center}

\vspace*{3cm}

% 試験区分
\fbox{\parbox{0.85\textwidth}{
  \centering
  \vspace{0.8em}
  {\Large 令和8年度}\\[1em]
  {\Huge\textbf{第1回 Arcaea共通テスト}}\\[1em]
  \vspace{0.8em}
}}

\vspace{3cm}

% 試験情報テーブル
\renewcommand{\arraystretch}{1.8}
\begin{tabular}{|>{\centering\arraybackslash}p{3.5cm}|>{\centering\arraybackslash}p{7cm}|}
\hline
\textbf{試験日} & 令和8年(2026年)1月 7日(日) \\
\hline
\textbf{解答時間(目安)} & 60分 \\
\hline
\textbf{出題科目} & Arcaea・文科・理科 \\
\hline
\end{tabular}

\vspace{4cm}

% 注意書き
\fbox{\parbox{0.75\textwidth}{
  \centering
  \vspace{0.5em}
  \textbf{注意事項}\\[0.5em]
  試験開始の合図があるまで,この冊子を開いてはいけません。\\[0.3em]
  \vspace{0.5em}
}}

\vspace{2cm}

% 主催
{\large 主催:Arcaea共通テスト製作チーム($\mathbb{X}$ : @Arcaea\_test)}

\end{center}

%===============================================
%  注意事項ページ
%===============================================
\newpage
\thispagestyle{instructionpage}

\begin{center}
{\LARGE\textbf{注 意 事 項}}\\[0.5em]
\rule{0.7\textwidth}{1pt}
\end{center}

\vspace{1.5em}

\section*{1 試験全般について}
\begin{enumerate}
  \item 解答時間の目安は60分であるが,あくまで目安である.これを超えて回答してもよい.
  \item 解答は,すべて定められたフォームに記入すること.
  \item 問題については,すべて Ver 6.11.0 時点の情報に基づいて解答するものとし,それ以降の仕様変更等によらない.
\end{enumerate}

\section*{2 解答の形式について}
\begin{enumerate}
  \item 解答に際しては,必ず各問いにおいて指定された解答形式に従うこと.
  たとえば,答えが $2.5$ となる場合であっても,$\dfrac{5}{2}$ と記すよう指示されているときは,その指示に従うこと.
  
  \item $\dfrac{10}{4}$ や $2\sqrt{8}$ など,約分されていないものや,根号が簡約されていない形での解答は,正解と認められない.
  
  \item 分数型での解答を求められた場合,分数の符号は分子に付けること.
  
  \item 小数型での解答を求められた場合,指定された桁数の一つ下の桁を四捨五入して答えること.
  たとえば,$\fbox{キ}.\fbox{ク}\fbox{ケ}$ に $2.495$ を解答する場合は,$2.50$ と記入すること.
  
  \item 問題文中の二重四角で示された \doublebox{ナ} などの欄には,選択肢の中から最も適切なものを一つ選んで答えること.
\end{enumerate}

\section*{3 禁止事項}
\begin{enumerate}
  \item 解答の作成にあたり,生成AIを使用してはならない.
  
  \item 解答に直接結びつく行為を目的としない範囲に限り,用語や背景事項の確認のための検索エンジンの使用を認める.
\end{enumerate}

\section*{4 その他}
\begin{enumerate}
  \item 試験終了後における,本試験の感想や雰囲気等についての発信を妨げるものではない.広く歓迎する.
\end{enumerate}

\vfill
\begin{center}
\rule{0.5\textwidth}{0.4pt}\\[0.5em]
\end{center}

%===============================================
%  Arcaea(科目開始例)
%===============================================
\beginsubject{Arcaea}
\rhead{Arcaea}
% ここに問題を記入
\daimon{第1問}
\answerset{0}% カウンタをリセット(アから開始)
\begin{enumerate}[label=(\arabic*)]
\item 次のうち,Arcaeaの登場人物としてふさわしくないものは\doublebox{a}である(コラボキャラを含む).
    \answergroup[2]{a}{%
    \begin{enumerate}[label=\textcircled{\scriptsize\arabic*}, itemsep=0pt, style=multiline, labelsep=0.5em]
    \item 光
    \item 紅
    \item マヤ
    \item サヤ
    \item 天下茶屋
    \end{enumerate}
    }

\item 次のうち,同じ楽曲名が複数存在するものは\doublebox{b}である(コラボ楽曲を含む).
    \answergroup[2]{b}{%
    \begin{enumerate}[label=\textcircled{\scriptsize\arabic*}, itemsep=0pt, style=multiline, labelsep=0.5em]
    \item Quon
    \item TEmPTaTiON
    \item GROLY : ROAD
    \item Jingle
    \end{enumerate}
    }
\end{enumerate}
% 例:
% \subsection*{第1問}
% 問題内容...

\vfill

%===============================================
%  文科
%===============================================
\beginsubject{文科}
\rhead{文科}
% ここに問題を記入


\vfill

%===============================================
%  理科
%===============================================
\beginsubject{理科}
\rhead{理科}
% ここに問題を記入
\daimon{第1問}
\begin{enumerate}[label=(\arabic*)]
  \item $7+6=\fbox{ア}\fbox{イ}$である.
\end{enumerate}
\daimon{第2問}

    データを効率よく管理する方法の一つとして,ハッシュ表と呼ばれる仕組みが用いられることがある.
    ハッシュ表では,各データに対応するキーの値から,そのデータを格納する位置を決めるために,ハッシュ関数と呼ばれる関数を用いる.
    ここでは,正の整数をキーとするデータを扱うこととし,キー $x$ に対するハッシュ関数 $h(x)$ を
    \[
    h(x)=x\mod n
    \]
    と定める.
    ここで,$n$ はハッシュ表の大きさを表す正の整数であり,$x\mod n$ は $x$ を $n$ で割った余りを表すものとする.
    異なるキーであっても,同じ値 $h(x)$ をとることがあり,このような場合を衝突という.
    以下の各問いに答えよ.
\begin{enumerate}[label=(\arabic*)]

  \item 自然数をキーとするデータを,ハッシュ表を用いて管理したい.
  キー$x$のハッシュ関数を$h(x)$と定めたとき,任意のキー$a$と$b$が衝突する条件は\doublebox{ウ}のときである.
  \answergroup{ウ}{%
    \begin{enumerate}[label=\textcircled{\scriptsize\arabic*}, itemsep=0pt, style=multiline, labelsep=0.5em]
    \item $a+bがnの倍数$
    \item $a-bがnの倍数$
    \item $nがa+bの倍数$
    \item $nがa-bの倍数$
    \end{enumerate}
    }
\end{enumerate}

% 表形式の解答欄サンプル(画像の再現)
\answertable{エ}{
  \begin{tabular}{|c|c|c|c|c|c|c|c|c|c|}
    \hline
     & \textcircled{\scriptsize 1} & \textcircled{\scriptsize 2} & \textcircled{\scriptsize 3} & \textcircled{\scriptsize 4} & \textcircled{\scriptsize 5} & \textcircled{\scriptsize 6} & \textcircled{\scriptsize 7} & \textcircled{\scriptsize 8} & \textcircled{\scriptsize 9} \\
    \hline
    ア & 正 & 正 & 正 & 0 & 0 & 0 & 負 & 負 & 負 \\
    \hline
    イ & 正 & 0 & 負 & 正 & 0 & 負 & 正 & 0 & 負 \\
    \hline
  \end{tabular}
}

\vfill
\finishsubject

%===============================================
%  謝辞
%===============================================
\newpage
\thispagestyle{acknowledgepage}

\begin{center}
{\LARGE\textbf{謝 辞}}\\[0.5em]
\rule{0.4\textwidth}{0.8pt}
\end{center}

\vspace{2em}

本試験の実施に際して,次の方々に深く感謝申し上げます(敬称略・順不同)。

\vspace{1.5em}

\begin{description}
  \item[全体] \textbf{Arcaea関係者の皆様}\\
   Arcaeaというゲームを世に出してくださいました。感謝の念に堪えません。

  \item[全体] \textbf{シュウ酸}(@TPA36497)\\
   本試験の問題を作成するにあたり,Arcaeaに収録されている楽曲についての情報をまとめていただきました。深謝の意を表します。

  \item[国語] \textbf{nkap}(@2354\_1423)\\
   のっく様のご紹介をいただきました。ありがとうございました。

  \item[国語] \textbf{のっく}(@nakkunarut23247)\\
  本試験の国語の問題を作成するにあたって,Arcaeaのストーリーに関しての考察を参考にすることを快諾していただきました.厚く御礼申し上げます。

  \item[全体] \textbf{受験者の皆様}\\
  本日は公私ご多用の中,本試験を受験してくださり,誠にありがとうございました。
\end{description}

\vfill

\begin{center}
\rule{0.3\textwidth}{0.4pt}\\[0.8em]
{\large 第1回 Arcaea共通テスト 終}
\end{center}

\end{document}